\documentclass[a4paper,13pt]{article}
\usepackage[T1]{fontenc}
\usepackage[utf8]{inputenc}
\usepackage[brazil]{babel}
\usepackage{listings}
\usepackage{minted}
\usepackage{graphicx}
\usepackage{indentfirst}

\begin{document}

01)Leia dois valores para as variáveis A e B, efetue a troca dos valores de forma que a variável A passe a ter o valor da variável B e que a variável B, o valor da variável A. Apresentar os valores trocados.
\\
\begin{minted}{bash}
#!/usr/bin/env bash

echo "Digite o valor da primeira variável: "
read VAR1

echo "Digite o valor da segunda variável: "
read VAR2

TROCA_VAR=$VAR1
VAR1=$VAR2
VAR2=$TROCA_VAR

echo "$VAR1  $VAR2"
\end{minted}
\\

02)Escreva um algoritmo que receba um número e mostre o número, se ele estiver entre quinze (inclusive) e quarenta.
\\
\begin{minted}{bash}
#!/usr/bin/env bash

echo "Por favor,digite um número: "
read NUM

if [ $NUM -gt 15 -a $NUM -lt 40 ];then
   echo "$NUM"
fi
\end{minted}
\\

03)Faça um algoritmo que receba nome, turma e três notas do aluno. Calcule a média ponderada considerando: primeira nota peso um, segunda nota peso dois e terceira nota peso três, informar o nome, a turma e a média do aluno que a média for inferior a sete.
\begin{minted}{bash}

#!/usr/bin/env bash

SOMA_NOTAS=0

echo "Digite o nome da(o) discente: "
read NOME

echo "Digite a turma: "
read TURMA

echo "Por favor,digite a primeira nota: "
read NOTA1

echo "Por favor,digite a segunda nota: "
read NOTA2

echo "Por favor,digite a terceira nota: "
read NOTA3

MEDIA=`echo "scale=4;$[($NOTA1 + $NOTA2*2 + $NOTA3*3)/6]" | bc -l`

echo "$NOME da $TURMA obteve média $MEDIA"
\end{minted}
\\

04)Construa um algoritmo que verifique se um número fornecido pelo usuário é primo ou não.

\\
\begin{minted}{bash}

 #!/usr/bin/env bash

echo "Por favor, digite um número: "
read NUM

if [ $NUM -eq 1 ];then
  echo "Não é primo"
fi

if [ $NUM -gt 1 ];then


if [ $NUM -gt 2  -a $[NUM%2] -eq 0 ];then
  echo "Não é primo"


else
  echo "É primo"
fi
fi
 
\end{minted}
\\

05)Faça um algoritmo que receba o salário de um funcionário, calcule e mostre o novo salário, sabendo-se que este sofreu um aumento de 25\%. Este aumento e válido para os funcionários com mais de cinco anos de serviço.
\\
\begin{minted}{bash}

#!/usr/bin/env bash

echo "Digite a quantidade de anos trabalho: "
read TEMP_SERV

echo "Digite o valor do salário: "
read SAL_ATUAL

if [ $TEMP_SERV -gt 5 ];then
  SAL_NOVO=`echo "escale=4;$SAL_ATUAL*1.25" | bc -l`
fi

echo "O salário reajustado é $SAL_NOVO"

\end{minted}

06)Entrar com um número e informar se ele é ou não divisível por 5.
\begin{minted}{bash}

#!/usr/bin/env bash

echo "Digite um número: "
read NUM

test $[NUM%5] -eq 0  && echo "Divisível por cinco" || echo "Não divisível por cinco"

\end{minted}

\\
07)Entrar com um número e informar se ele é divisível por 3 e por 7.\\
\begin{minted}{bash}
#!/usr/bin/env bash

echo "Por favor, digite um número: "
read NUM

[ $[NUM%3] -eq 0 -a $[NUM%7] -eq 0 ] && echo "O número é divisível por 3 e 7"
\end{minted}
\\
08)Um funcionário irá receber um aumento de acordo com o seu plano de trabalho, de acordo com a tabela abaixo\\
\begin{figure}[H]
 \centering
 \includegraphics[width=7cm]{imagens/f2.png}
\end{figure}

\begin{minted}{bash}
#!/usr/bin/env bash

echo "Por favor, digite o plano A,B ou C: "
read PLANO

echo "Por favor, digite o valor do salário: "
read SAL_ATUAL

case $PLANO in
"A")
 NOVO_SALARIO=`echo "scale=4;$SAL_ATUAL*1.10" | bc -l`
 echo "O salário reajustado  é $NOVO_SALARIO"
 ;;

"B")
 NOVO_SALARIO=`echo "scale=4;$SAL_ATUAL*1.15" | bc -l`
 echo "O salário reajustado  é $NOVO_SALARIO"
 ;;

"C")
  NOVO_SALARIO=`echo "scale=4;$SAL_ATUAL*1.20" | bc -l`
  echo "O salário reajustado  é $NOVO_SALARIO"
 ;;
 
 *) echo "Errrrrrrrrou";;
 
esac
\end{minted}
\\
09)O cardápio do lanche Pão com Ovo é o seguinte\\
\begin{figure}[htb]
 \centering
 \includegraphics[width=7cm]{imagens/f1.png}
\end{figure}
\\
Implemente um programa que leia o código do item pedido, a quantidade e calcule o valor a ser pago por aquele lanche. Considere que a cada execução somente será calculado um item.\\
\begin{minted}{bash}
#!/usr/bin/env bash

echo "Por obséquio, digite o código do pedido(100 até 105) :"
read COD_PED

echo "Agora, a quantidade desejada :"
read QUANT

case $COD_PED in

 100)
    VALOR_PAGAR=`echo "scale=5;$QUANT*1.2" | bc -l`
    echo "O valor devido é $VALOR_PAGAR"
    ;;
 101) 
    VALOR_PAGAR=`echo "scale=5;$QUANT*1.30" | bc -l`
    echo "O valor devido é $VALOR_PAGAR"
    ;;
 
 102)
   VALOR_PAGAR=`echo "scale=5;$QUANT*1.50" | bc -l`
    echo "O valor devido é $VALOR_PAGAR"
    ;;
 
 103)
   VALOR_PAGAR=`echo "scale=5;$QUANT*1.20" | bc -l`
   echo "O valor devido é $VALOR_PAGAR"
   ;;
 
 104)
  VALOR_PAGAR=`echo "scale=5;$QUANT*1.30" | bc -l`
  echo "O valor devido é $VALOR_PAGAR"
  ;; 
 
 105)
   VALOR_PAGAR=`echo "scale=5;$QUANT*1.0" | bc -l`
   echo "O valor devido é $VALOR_PAGAR"
   ;;
  
 *) echo "Código inserido é inválido!";;
 
 esac

\end{minted}
\\
10)Escreva um algoritmo que receba dois números e informe a diferença do maior
pelo menor.
\\
\begin{minted}{bash}

#!/usr/bin/env bash

echo "Digite um número: "
read NUM1

echo "Digite outro número: "
read NUM2

if [ $NUM1 -gt $NUM2 ];then
  diferenca=$(($NUM1-$NUM2))

 else
   diferenca=$(($NUM2-$NUM1))
fi

echo "$diferenca"
\end{minted}

\end{document}
